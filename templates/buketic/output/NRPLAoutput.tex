% \documentclass{article}
% \usepackage[utf8]{inputenc}
% \usepackage[russian]{babel}
% \usepackage{geometry}
% \usepackage{titlesec}

% \begin{document}
% Привет, ! Уже  год
% \end{document}

\documentclass[a4paper,11pt]{article}
\usepackage[utf8]{inputenc}
\usepackage[T1]{fontenc}
\usepackage[russian]{babel}
\usepackage{geometry}
\usepackage{titlesec}

\geometry{top=20mm, left=20mm, right=20mm, bottom=20mm}

% Настройка стиля заголовков
\titleformat{\section}{\normalfont\large\bfseries}{\thesection.}{1em}{\centering}

\begin{document}

\begin{center}
\textbf{\large ДОГОВОР АРЕНДЫ НЕЖИЛОГО ПОМЕЩЕНИЯ №  }
\end{center}

\vspace{2mm}

\noindent
г. \underline{ \textbf{  } }  \hfill
\underline{ \textbf{  } г.} \\
\makebox[0.5cm]{\hfill} (город) 

\vspace{1cm}


\vspace{2mm}

\noindent
ООО \textbf{ <<>> } в лице \textbf{  }, действующего на основании
\textbf{  } (далее — «Арендодатель»), с одной стороны, и \\
ООО \textbf{ <<>> } в лице \textbf{  }, действующего на основании
\textbf{  } (далее — «Арендатор»), с другой стороны, \\
заключили настоящий договор о нижеследующем:


\section*{1. ПРЕДМЕТ ДОГОВОРА}

\noindent
1.1. Арендодатель сдает Арендатору в аренду нежилое помещение площадью \textbf{  } кв. м, в том числе торговая площадь \textbf{  } кв. м, расположенное по адресу: \textbf{ Москва }.

\noindent
1.2. Арендатор принимает указанное в п. 1.1. договорное нежилое помещение для осуществления своей хозяйственной деятельности.

\noindent
1.3. Переданное в аренду строение является собственностью Арендодателя, результаты хозяйственной деятельности Арендатора с использованием арендованного строения принадлежат Арендатору.

\noindent
Отделимые улучшения арендованного строения, осуществленные за счет собственных средств Арендатора, являются его собственностью.

\noindent
Неотделимые улучшения арендованного строения после прекращения действия настоящего договора безвозмездно переходят в собственность Арендодателя.


\section*{2. ПРАВА И ОБЯЗАННОСТИ СТОРОН}

\noindent
2.1. Арендатору предоставляется право контролировать исполнение Арендатором условий договора.

\noindent
2.2. Арендатор обеспечивает сохранность принятого в аренду помещения, за свой счет производит текущий и капитальный ремонт, ремонт инженерных сетей, ЭСО и государственного пожарного надзора.

\noindent
2.3. Сдача арендованного помещения в субаренду производится только Арендодателем по соглашению с Арендатором.

\noindent
2.4. Арендатор обязан соблюдать условия хозяйственной эксплуатации арендованного помещения.

\noindent
2.5. Арендатор имеет право производить перепланировку и переоборудование арендованного помещения только при наличии на то письменного согласия Арендодателя, если такая перепланировка или переоборудование не ухудшает техническое состояние элементов и конструкций помещения.

\noindent
2.6. За пределами обязательств по настоящему договору Арендатор свободен в осуществлении своей хозяйственной деятельности.

\section*{3. РАСЧЕТЫ ПО ДОГОВОРУ}

\noindent
3.1. За аренду указанного в п.1.1. договора помещения Арендатор перечисляет арендную плату с учетом (без учета пользования коммунальными услугами и электроэнергией) в размере \textbf{  } рублей.

\noindent
3.2. Перечисление арендной платы производится ежемесячно в срок до \textbf{  } числа месяца, следующего за расчетным.

\vfill

\noindent
\underline{\hspace{5cm}} \hspace{6.7cm} \underline{\hspace{5cm}} \\
\textit{(Подпись Арендодателя)} \hspace{8cm} \textit{(Подпись Арендатора)}

\end{document}